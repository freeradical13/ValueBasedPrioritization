\title{Value-Based Prioritization\cite{SourceCode}}

\author{Kevin Grigorenko\cite{Author1}}

\newcommand{\abstractText}{\noindent
A method is proposed to use value theory to quantitatively prioritize
potential actions to accomplish a goal. This method is applied to the
example of choosing meaningful work using an example value system based
on the desire to reduce suffering.
}

%%%%%%%%%%%%%%%%%
% Configuration %
%%%%%%%%%%%%%%%%%

\documentclass[12pt, a4paper, twocolumn]{article}
\usepackage{amsmath}
\usepackage{amssymb}
\usepackage{mathtools}
\usepackage{xurl}
\usepackage{hyperref}
\hypersetup{colorlinks=true, urlcolor=blue, linkcolor=blue, citecolor=blue}
\usepackage[super,comma,sort&compress]{natbib}
\usepackage{abstract}
\renewcommand{\abstractnamefont}{\normalfont\bfseries}
\renewcommand{\abstracttextfont}{\normalfont\itshape}
\usepackage{lipsum}
\usepackage{geometry}
\geometry{top=1cm,bottom=1.5cm,left=2cm,right=2cm,includehead,includefoot}
\setlength{\columnsep}{7mm} % Column separation width
\renewcommand*{\thefootnote}{\fnsymbol{footnote}} % Use symbols for footnotes
\renewcommand*{\bibfont}{\raggedright}

%%%%%%%%%%%%%%
% References %
%%%%%%%%%%%%%%

\begin{filecontents}{value_based_prioritization.bib}

@misc{SourceCode,
  author       = {Grigorenko, Kevin},
  title        = {Value-Based Prioritization: Source Code},
  howpublished = {\url{https://github.com/freeradical13/ValueBasedPrioritization}},
}

@misc{Author1,
  author       = {Grigorenko, Kevin},
  howpublished = {\url{mailto:kevin@myplaceonline.com}},
}

@InCollection{scientific-method,
  author       = {Andersen, Hanne and Hepburn, Brian},
  title        = {Scientific Method},
  booktitle    = {The Stanford Encyclopedia of Philosophy},
  editor       = {Edward N. Zalta},
  note         = {\url{https://plato.stanford.edu/archives/sum2016/entries/scientific-method/}},
  year         = {2016},
  edition      = {Summer 2016},
  publisher    = {Metaphysics Research Lab, Stanford University}
}

@article{martela2016three,
  title        = {The three meanings of meaning in life: Distinguishing coherence, purpose, and significance},
  author       = {Martela, Frank and Steger, Michael F},
  journal      = {The Journal of Positive Psychology},
  volume       = {11},
  number       = {5},
  pages        = {531--545},
  year         = {2016},
  publisher    = {Taylor \& Francis},
  note         = {\url{https://dx.doi.org/10.1080/17439760.2015.1137623}}
}

@book{huemer2007ethical,
  title        = {Ethical intuitionism},
  author       = {Huemer, Michael},
  year         = {2007},
  publisher    = {Springer}
}

@incollection{huemer2013problem,
  title        = {The problem of political authority},
  author       = {Huemer, Michael},
  booktitle    = {The Problem of Political Authority},
  pages        = {3--19},
  year         = {2013},
  publisher    = {Springer}
}

@InCollection{value-theory,
  author       = {Schroeder, Mark},
  title        = {Value Theory},
  booktitle    = {The Stanford Encyclopedia of Philosophy},
  editor       = {Edward N. Zalta},
  note         = {\url{https://plato.stanford.edu/archives/fall2016/entries/value-theory/}},
  year         = {2016},
  edition      = {Fall 2016},
  publisher    = {Metaphysics Research Lab, Stanford University}
}

@InCollection{consequentialism,
  author       = {Sinnott-Armstrong, Walter},
  title        = {Consequentialism},
  booktitle    = {The Stanford Encyclopedia of Philosophy},
  editor       = {Edward N. Zalta},
  note         = {\url{https://plato.stanford.edu/archives/win2015/entries/consequentialism/}},
  year         = {2015},
  edition      = {Winter 2015},
  publisher    = {Metaphysics Research Lab, Stanford University}
}

@InCollection{ayn-rand,
  author       = {Badhwar, Neera K. and Long, Roderick T.},
  title        = {Ayn Rand},
  booktitle    = {The Stanford Encyclopedia of Philosophy},
  editor       = {Edward N. Zalta},
  note         = {\url{https://plato.stanford.edu/archives/fall2017/entries/ayn-rand/}},
  year         = {2017},
  edition      = {Fall 2017},
  publisher    = {Metaphysics Research Lab, Stanford University}
}

@InCollection{religion-morality,
  author       = {Hare, John},
  title        = {Religion and Morality},
  booktitle    = {The Stanford Encyclopedia of Philosophy},
  editor       = {Edward N. Zalta},
  note         = {\url{https://plato.stanford.edu/archives/win2014/entries/religion-morality/}},
  year         = {2014},
  edition      = {Winter 2014},
  publisher    = {Metaphysics Research Lab, Stanford University}
}

@InCollection{morality-biology,
  author       = {FitzPatrick, William},
  title        = {Morality and Evolutionary Biology},
  booktitle    = {The Stanford Encyclopedia of Philosophy},
  editor       = {Edward N. Zalta},
  note         = {\url{https://plato.stanford.edu/archives/spr2016/entries/morality-biology/}},
  year         = {2016},
  edition      = {Spring 2016},
  publisher    = {Metaphysics Research Lab, Stanford University}
}

@InCollection{epicurus,
  author       = {Konstan, David},
  title        = {Epicurus},
  booktitle    = {The Stanford Encyclopedia of Philosophy},
  editor       = {Edward N. Zalta},
  note         = {\url{https://plato.stanford.edu/archives/sum2018/entries/epicurus/}},
  year         = {2018},
  edition      = {Summer 2018},
  publisher    = {Metaphysics Research Lab, Stanford University}
}

@InCollection{stoicism,
  author       = {Baltzly, Dirk},
  title        = {Stoicism},
  booktitle    = {The Stanford Encyclopedia of Philosophy},
  editor       = {Edward N. Zalta},
  note         = {\url{https://plato.stanford.edu/archives/sum2018/entries/stoicism/}},
  year         = {2018},
  edition      = {Summer 2018},
  publisher    = {Metaphysics Research Lab, Stanford University}
}

@InCollection{rawls,
  author       = {Wenar, Leif},
  title        = {John Rawls},
  booktitle    = {The Stanford Encyclopedia of Philosophy},
  editor       = {Edward N. Zalta},
  howpublished = {\url{https://plato.stanford.edu/archives/spr2017/entries/rawls/}},
  year         = {2017},
  edition      = {Spring 2017},
  publisher    = {Metaphysics Research Lab, Stanford University}
}

@InCollection{communitarianism,
  author       = {Bell, Daniel},
  title        = {Communitarianism},
  booktitle    = {The Stanford Encyclopedia of Philosophy},
  editor       = {Edward N. Zalta},
  howpublished = {\url{https://plato.stanford.edu/archives/sum2016/entries/communitarianism/}},
  year         = {2016},
  edition      = {Summer 2016},
  publisher    = {Metaphysics Research Lab, Stanford University}
}

\end{filecontents}

\begin{document}

%%%%%%%%%%%%
% Abstract %
%%%%%%%%%%%%

\twocolumn[
  \begin{@twocolumnfalse}
    \maketitle
    \begin{abstract}
      \abstractText
      \newline
      \newline
    \end{abstract}
  \end{@twocolumnfalse}
]

%%%%%%%%%%%
% Article %
%%%%%%%%%%%

\section{Introduction}

Why should a particular goal be pursued (``Why")? Given a goal, what actions should be pursued to best accomplish said goal (``What")? Given an action, how should said action be pursued (``How")?

This article proposes that value theory usually best scopes ``Why" and ``What" and the scientific method usually best answers ``How". A method called Value-Based Prioritization \eqref{article-equation} is developed to answer the ``What" question:

\begin{equation}\label{article-equation}
  \begin{gathered}
    \textrm{Why: } \textit{Value Theory} \\
    \downarrow \\
    \textrm{What: } \textit{\textbf{Value-Based Prioritization}} \\
    \downarrow \\
    \textrm{How: } \textit{Scientific Method}
  \end{gathered}
\end{equation}

\section{Why a Goal?}

``Why a Goal?" is usually best scoped using a value system because value systems are evaluative by nature\cite{value-theory}. Evaluating different value systems is left as an (lifelong) exercise for the reader\footnote{Example value systems include intuitionism\cite{huemer2007ethical}, consequentialism\cite{consequentialism}, evolutionary biology\cite{morality-biology}, religion\cite{religion-morality}, epicureanism\cite{epicurus}, stoicism\cite{stoicism}, political liberalism\cite{rawls}, anarcho-capitalism\cite{huemer2013problem}, communitarianism\cite{communitarianism}, objectivism\cite{ayn-rand}, etc.}.

\section{What Actions?}

``What Actions?" is usually best scoped by prioritizing actions because actions usually have differing effect sizes and time is limited. It follows from the value system used to answer ``Why" that the same value system is used primarily to evaluate the priority of each action.

This article proposes a method called Value-Based Prioritization which builds a quantitative prioritization model based on predicted effect sizes. Raw prioritization scores are further scaled by contextual factors such as implementation time, cost, risk, and other judgments.

\section{How to do an Action?}

Given answers to ``Why?" and ``What?", how to implement actions is usually best answered with the scientific method\cite{scientific-method}: observations are made and rational thought is used to generate hypotheses, hypotheses are tested with experiments, and successful experiments lead to theories and results.

\section{Value-Based Prioritization}

A \textbf{value system} $V$ generates a \textbf{goal} $G(t)$ (for some future time $t$) and a set of \textbf{mutually exclusive potential future actions} $A(t) = \{A_1(t), \ldots, A_N(t)\}$.

An action's \textbf{estimated relative accomplishment amount} $B(A(t))$ \eqref{action-amount} is an action's expected \textit{relative} (i.e. with respect to other actions) contribution towards accomplishing $G$(t):

\begin{equation}\label{action-amount}
  \begin{gathered}
B(A(t)) = \mathbb{R}, \\
0 \leq \mathbb{R} \leq 1
  \end{gathered}
\end{equation}

Thus, $G(t)$ is fully accomplished if all actions are accomplished:

\begin{equation}\label{goal-accomplished}
G(t) = \sum_{i=1}^{N} B(A_i(t)) = 1
\end{equation}

A \textbf{value-based prioritization score} $C(A(t))$ \eqref{prioritization-score} is the result of the product of a set of \textbf{value-based prioritization scale functions} $S = \{S_1, \ldots, S_N\}$ multiplied by \eqref{action-amount}:

\begin{equation}\label{prioritization-score}
  \begin{gathered}
C(A(t)) = B(A(t)) \cdot \prod_{j=1}^{N} S_j(A(t)), \\
0 \leq S_j(B(A(t))) \leq 1
  \end{gathered}
\end{equation}

Example scale functions include cost, risk, etc. The set $S$ always includes the element $S_0(A(t)) = 1$. Note that $\sum_{i=1}^{N} C(A_i(t)) \neq G$ if any $S_j(A_i(t)) < 1$.

% \begin{itemize}
% \item Implementation time
% \item Cost
% \item Risk
% \item Likelihood of success
% \item Plays into Strengths
% \item Achieves passions/environmental needs
% \item Achieves flow
% \item Supports life and family
% \item Return on investment
% \item Initialization time (learning, starting, etc.)
% \item Interest
% \item Market demand
% \end{itemize}

A \textbf{value-based prioritization} $Z(t)$ \eqref{vbp} is a sequence of actions ordered by prioritization score \eqref{prioritization-score} in descending order:

\begin{equation}\label{vbp}
  \begin{gathered}
Z(t) = (A_1(t), \ldots, A_N(t)), \\
C(A_1(t)) \geq \ldots \geq C(A_N(t))
  \end{gathered}
\end{equation}

The first $j$ actions in $Z(t)$ should be executed in descending priority/proportion where $j$ is chosen based on available concurrency.

\section{Modeled Value-Based Prioritization}

Historical data may be used to predict actions' estimated relative accomplishment amounts \eqref{action-amount}.

If each action has historical data $D(A)$ \eqref{action-predictors}:

\begin{equation}\label{action-predictors}
\begin{multlined}
D(A) = \\
\shoveleft[0.5cm]{((t_1, D(A,t_1)), \ldots, (t_N, D(A,t_N)))}
\end{multlined}
\end{equation}

Then, a set of \textbf{comparable prediction models} $R(D(A))$ \eqref{prediction-models} is applied to each $D(A)$ (e.g. linear regression) :

\begin{equation}\label{prediction-models}
\begin{multlined}
R(D(A)) =\\
\{R_1(D(A)), \ldots, R_N(D(A))\}
\end{multlined}
\end{equation}

The models are compared (e.g. using ANOVA) and the \textbf{best fitting model} $M(R(D(A)))$ is selected. Note that $M(R(D(A)))$ may be different for each action.

Next, the model $M(R(D(A)))$ is used to predict each $B(A(t_F))$ for some time in the future $t_F$ (e.g. the average time actions will take to ramp up implementation).

Finally, value-based prioritization \eqref{vbp} is run with $Z(t_F)$.

%%%%%%%%%%%%%%
% References %
%%%%%%%%%%%%%%

\nocite{*}
\bibliographystyle{plainnat}
\bibliography{value_based_prioritization}

\end{document}

% Create PDF on Linux:
% FILE=value_based_prioritization; pkill -9 -f ${FILE} &>/dev/null; rm -f ${FILE}*aux ${FILE}*bbl ${FILE}*bib ${FILE}*blg ${FILE}*log ${FILE}*out ${FILE}*pdf &>/dev/null; pdflatex -halt-on-error ${FILE}; bibtex ${FILE} && pdflatex ${FILE} && pdflatex ${FILE} && (xdg-open ${FILE}.pdf &)
