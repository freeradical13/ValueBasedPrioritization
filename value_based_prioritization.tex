%%%%%%%%%%%%%%%%%
% Configuration %
%%%%%%%%%%%%%%%%%

\documentclass[12pt, a4paper, twocolumn]{article}
\usepackage{amsmath}
\usepackage{amssymb}
\usepackage{mathtools}
\usepackage{xurl}
\usepackage{hyperref}
\hypersetup{colorlinks=true, urlcolor=blue, linkcolor=blue, citecolor=blue}
\urlstyle{rm}
\usepackage[super,comma,sort&compress]{natbib}
\bibliographystyle{plainnat}
\usepackage{abstract}
\renewcommand{\abstractnamefont}{\normalfont\bfseries}
\renewcommand{\abstracttextfont}{\normalfont\itshape}
\usepackage{lipsum}
\usepackage{booktabs}
\usepackage{geometry}
\geometry{top=1cm,bottom=1.5cm,left=2cm,right=2cm,includehead,includefoot}
\setlength{\columnsep}{7mm} % Column separation width
\renewcommand*{\thefootnote}{\roman{footnote}} % Use symbols for footnotes
\renewcommand*{\bibfont}{\raggedright}

% Inline equations: https://tex.stackexchange.com/a/78582/178803
\makeatletter
\newcommand*{\inlineequation}[2][]{%
  \begingroup
    % Put \refstepcounter at the beginning, because
    % package `hyperref' sets the anchor here.
    \refstepcounter{equation}%
    \ifx\\#1\\%
    \else
      \label{#1}%
    \fi
    % prevent line breaks inside equation
    \relpenalty=10000 %
    \binoppenalty=10000 %
    \ensuremath{%
      % \displaystyle % larger fractions, ...
      #2%
    }%
    ~\@eqnnum
  \endgroup
}
\makeatother

%%%%%%%%%%%%%%
% References %
%%%%%%%%%%%%%%

\begin{filecontents}{value_based_prioritization.bib}

@InCollection{scientific-method,
  author       = {Andersen, Hanne and Hepburn, Brian},
  title        = {Scientific Method},
  booktitle    = {The Stanford Encyclopedia of Philosophy},
  editor       = {Edward N. Zalta},
  note         = {\url{https://plato.stanford.edu/archives/sum2016/entries/scientific-method/}},
  year         = {2016},
  edition      = {Summer 2016},
  publisher    = {Metaphysics Research Lab, Stanford University}
}

@article{martela2016three,
  title        = {The three meanings of meaning in life: Distinguishing coherence, purpose, and significance},
  author       = {Martela, Frank and Steger, Michael F},
  journal      = {The Journal of Positive Psychology},
  volume       = {11},
  number       = {5},
  pages        = {531--545},
  year         = {2016},
  publisher    = {Taylor \& Francis},
  note         = {\url{https://dx.doi.org/10.1080/17439760.2015.1137623}}
}

@book{huemer2007ethical,
  title        = {Ethical Intuitionism},
  author       = {Huemer, Michael},
  year         = {2007},
  publisher    = {Springer}
}

@book{huemer2013problem,
  title        = {The Problem of Political Authority},
  author       = {Huemer, Michael},
  year         = {2013},
  publisher    = {Springer}
}

@InCollection{value-theory,
  author       = {Schroeder, Mark},
  title        = {Value Theory},
  booktitle    = {The Stanford Encyclopedia of Philosophy},
  editor       = {Edward N. Zalta},
  note         = {\url{https://plato.stanford.edu/archives/fall2016/entries/value-theory/}},
  year         = {2016},
  edition      = {Fall 2016},
  publisher    = {Metaphysics Research Lab, Stanford University}
}

@InCollection{consequentialism,
  author       = {Sinnott-Armstrong, Walter},
  title        = {Consequentialism},
  booktitle    = {The Stanford Encyclopedia of Philosophy},
  editor       = {Edward N. Zalta},
  note         = {\url{https://plato.stanford.edu/archives/win2015/entries/consequentialism/}},
  year         = {2015},
  edition      = {Winter 2015},
  publisher    = {Metaphysics Research Lab, Stanford University}
}

@InCollection{ayn-rand,
  author       = {Badhwar, Neera K. and Long, Roderick T.},
  title        = {Ayn Rand},
  booktitle    = {The Stanford Encyclopedia of Philosophy},
  editor       = {Edward N. Zalta},
  note         = {\url{https://plato.stanford.edu/archives/fall2017/entries/ayn-rand/}},
  year         = {2017},
  edition      = {Fall 2017},
  publisher    = {Metaphysics Research Lab, Stanford University}
}

@InCollection{religion-morality,
  author       = {Hare, John},
  title        = {Religion and Morality},
  booktitle    = {The Stanford Encyclopedia of Philosophy},
  editor       = {Edward N. Zalta},
  note         = {\url{https://plato.stanford.edu/archives/win2014/entries/religion-morality/}},
  year         = {2014},
  edition      = {Winter 2014},
  publisher    = {Metaphysics Research Lab, Stanford University}
}

@InCollection{morality-biology,
  author       = {FitzPatrick, William},
  title        = {Morality and Evolutionary Biology},
  booktitle    = {The Stanford Encyclopedia of Philosophy},
  editor       = {Edward N. Zalta},
  note         = {\url{https://plato.stanford.edu/archives/spr2016/entries/morality-biology/}},
  year         = {2016},
  edition      = {Spring 2016},
  publisher    = {Metaphysics Research Lab, Stanford University}
}

@InCollection{epicurus,
  author       = {Konstan, David},
  title        = {Epicurus},
  booktitle    = {The Stanford Encyclopedia of Philosophy},
  editor       = {Edward N. Zalta},
  note         = {\url{https://plato.stanford.edu/archives/sum2018/entries/epicurus/}},
  year         = {2018},
  edition      = {Summer 2018},
  publisher    = {Metaphysics Research Lab, Stanford University}
}

@InCollection{stoicism,
  author       = {Baltzly, Dirk},
  title        = {Stoicism},
  booktitle    = {The Stanford Encyclopedia of Philosophy},
  editor       = {Edward N. Zalta},
  note         = {\url{https://plato.stanford.edu/archives/sum2018/entries/stoicism/}},
  year         = {2018},
  edition      = {Summer 2018},
  publisher    = {Metaphysics Research Lab, Stanford University}
}

@InCollection{rawls,
  author       = {Wenar, Leif},
  title        = {John Rawls},
  booktitle    = {The Stanford Encyclopedia of Philosophy},
  editor       = {Edward N. Zalta},
  note         = {\url{https://plato.stanford.edu/archives/spr2017/entries/rawls/}},
  year         = {2017},
  edition      = {Spring 2017},
  publisher    = {Metaphysics Research Lab, Stanford University}
}

@InCollection{communitarianism,
  author       = {Bell, Daniel},
  title        = {Communitarianism},
  booktitle    = {The Stanford Encyclopedia of Philosophy},
  editor       = {Edward N. Zalta},
  note         = {\url{https://plato.stanford.edu/archives/sum2016/entries/communitarianism/}},
  year         = {2016},
  edition      = {Summer 2016},
  publisher    = {Metaphysics Research Lab, Stanford University}
}

@article{weinstein2009qalys,
  title        = {QALYs: the basics},
  author       = {Weinstein, Milton C and Torrance, George and McGuire, Alistair},
  journal      = {Value in health},
  volume       = {12},
  pages        = {S5--S9},
  year         = {2009},
  publisher    = {Wiley Online Library},
  note         = {\url{https://onlinelibrary.wiley.com/doi/pdf/10.1111/j.1524-4733.2009.00515.x}}
}

@article{zucchini2000introduction,
  title        = {An introduction to model selection},
  author       = {Zucchini, Walter},
  journal      = {Journal of mathematical psychology},
  volume       = {44},
  number       = {1},
  pages        = {41--61},
  year         = {2000},
  publisher    = {Academic Press},
  note         = {\url{http://www.indiana.edu/~clcl/Q550/Papers/Zucchini_JMP_2000.pdf}}
}

@article{wit2012all,
  title        = {‘All models are wrong...’: an introduction to model uncertainty},
  author       = {Wit, Ernst and Heuvel, Edwin van den and Romeijn, Jan-Willem},
  journal      = {Statistica Neerlandica},
  volume       = {66},
  number       = {3},
  pages        = {217--236},
  year         = {2012},
  publisher    = {Wiley Online Library},
  note         = {\url{https://www.rug.nl/research/portal/files/13270992/2012StatistNeerlWit.pdf}}
}

@misc{centers2017underlying,
  title        = {\textit{Underlying Cause of Death 1999-2017 on CDC WONDER Online Database, released December, 2018. Data are from the Multiple Cause of Death Files, 1999-2017, as compiled from data provided by the 57 vital statistics jurisdictions through the Vital Statistics Cooperative Program}},
  author       = {{Centers for Disease Control and Prevention and National Center for Health Statistics}},
  howpublished = {\url{https://wonder.cdc.gov/ucd-icd10.html}},
  note         = {Accessed: 2019-01-31},
}

@book{icd10,
  title={International statistical classification of diseases and related health problems},
  author={World Health Organization},
  year={2016},
  publisher={World Health Organization},
  edition={10th},
  note={\url{https://apps.who.int/iris/bitstream/handle/10665/246208/9789241549165-V1-eng.pdf}}
}

\end{filecontents}

\begin{document}

\title{Value-Based Prioritization\thanks{\url{https://github.com/freeradical13/ValueBasedPrioritization}}}

\author{Kevin Grigorenko\thanks{\href{mailto:kevin@myplaceonline.com}{kevin@myplaceonline.com}}}

\newcommand{\abstractText}{\noindent
A method is proposed to use value theory to quantitatively prioritize
potential actions to accomplish a goal. This method is applied to the
example of choosing meaningful work using an example value system based
on the desire to reduce suffering.
}

%%%%%%%%%%%%
% Abstract %
%%%%%%%%%%%%

\twocolumn[
  \begin{@twocolumnfalse}
    \maketitle
    \begin{abstract}
      \abstractText
      \newline
      \newline
    \end{abstract}
  \end{@twocolumnfalse}
]

\saythanks

%%%%%%%%%%%
% Article %
%%%%%%%%%%%

\section{Introduction}

Why should a particular goal be pursued (``Why")? Given a goal, what actions should be pursued to best accomplish said goal (``What")? Given an action, how should said action be pursued (``How")?

This article proposes that value theory usually best scopes ``Why" and ``What" and the scientific method usually best answers ``How". A method called Value-Based Prioritization is developed to answer the ``What" question:

\begin{equation}\label{article-equation}
  \begin{gathered}
    \textrm{Why: } \textit{Value Theory} \\
    \downarrow \\
    \textrm{What: } \textit{\textbf{Value-Based Prioritization}} \\
    \downarrow \\
    \textrm{How: } \textit{Scientific Method}
  \end{gathered}
\end{equation}

\section{Why a Goal?}

``Why a Goal?" is usually best scoped using value systems because they are evaluative by nature\cite{value-theory}. Evaluating different value systems is left as an (lifelong) exercise for the reader\footnote{Example value systems include intuitionism\cite{huemer2007ethical}, consequentialism\cite{consequentialism}, evolutionary biology\cite{morality-biology}, religion\cite{religion-morality}, epicureanism\cite{epicurus}, stoicism\cite{stoicism}, political liberalism\cite{rawls}, anarcho-capitalism\cite{huemer2013problem}, communitarianism\cite{communitarianism}, objectivism\cite{ayn-rand}, etc.}.

\section{What Actions?}

``What Actions?" is usually best scoped by prioritizing actions because actions usually have differing effect sizes and time is limited. It follows from the value system used to answer ``Why" that the same value system is used primarily to evaluate the priority of each action.

This article proposes a method called Value-Based Prioritization which builds a quantitative prioritization model based on predicted effect sizes. Raw prioritization scores are further scaled by contextual factors such as implementation time, cost, risk, and other judgments.

\section{How to do an Action?}

Given answers to ``Why?" and ``What?", how to implement actions is usually best answered with the scientific method\cite{scientific-method}: observations are made and rational thought is used to generate hypotheses, hypotheses are tested with experiments, and successful experiments lead to theories and results.

\section{Value-Based Prioritization}

A \textbf{value system} \inlineequation[value-system]{V} generates a \textbf{goal} \inlineequation[goal]{G(t)} (for some future time $t$) and a set of \textbf{mutually exclusive potential future actions} $A(t)$:

\begin{equation}\label{potential-actions}
  \begin{gathered}
A(t) = \{A_1(t), \ldots, A_N(t)\}, \\
N > 1
  \end{gathered}
\end{equation}

An action's \textbf{estimated relative accomplishment amount} $B(A(t))$ is an action's expected \textit{relative} (i.e. with respect to other actions) contribution towards accomplishing $G$(t):

\begin{equation}\label{action-amount}
  \begin{gathered}
B(A(t)) = \mathbb{R}, \\
0 \leq \mathbb{R} \leq 1
  \end{gathered}
\end{equation}

Thus, $G(t)$ is fully accomplished if all actions are accomplished:

\begin{equation}\label{goal-accomplished}
G(t) = \sum_{i=1}^{N} B(A_i(t)) = 1
\end{equation}

A \textbf{value-based prioritization score} $C(A(t))$ is the result of the product of a set of \textbf{value-based prioritization scale functions} \inlineequation[scale-functions]{S = \{S_1, \ldots, S_N\}} multiplied by \eqref{action-amount}:

\begin{equation}\label{prioritization-score}
  \begin{gathered}
C(A(t)) = B(A(t)) \cdot \prod_{j=1}^{N} S_j(A(t)), \\
0 \leq S_j(B(A(t))) \leq 1
  \end{gathered}
\end{equation}

Example scale functions include implementation time, cost, risk, and other judgments. Ideally, scale functions should be defined before running the model to reduce bias. The set $S$ always includes the element $S_1(A(t)) = 1$. Note that $\sum_{i=1}^{N} C(A_i(t)) \neq G$ if any $S_j(A_i(t)) < 1$.

A \textbf{value-based prioritization} $Z(t)$ is a sequence of actions ordered by prioritization score \eqref{prioritization-score} in descending order:

\begin{equation}\label{vbp}
  \begin{gathered}
Z(t) = (A_1(t), \ldots, A_N(t)), \\
C(A_1(t)) \geq \ldots \geq C(A_N(t))
  \end{gathered}
\end{equation}

The first $k$ actions in $Z(t)$ should be executed in descending priority/proportion where \inlineequation[how-many-actions]{k} is chosen based on factors such as available concurrency, time, resources, etc.

\section{Modeled Value-Based Prioritization}

Historical data may be used to predict actions' estimated relative accomplishment amounts \eqref{action-amount} at a future time \inlineequation[modeled-future-time]{t_F}  (e.g. the average time actions will take to ramp up implementation).

If each action has historical data $D(A)$ \eqref{action-predictors}:

\begin{equation}\label{action-predictors}
\begin{multlined}
D(A) = \\
\shoveleft[0.5cm]{((t_1, D(A,t_1)), \ldots, (t_N, D(A,t_N)))}
\end{multlined}
\end{equation}

Then, a set of \textbf{comparable prediction models} $R(D(A))$ \eqref{prediction-models} is applied to each $D(A)$ (e.g. linear regression with different degrees):

\begin{equation}\label{prediction-models}
\begin{multlined}
R(D(A)) =\\
\{R_1(D(A)), \ldots, R_N(D(A))\}
\end{multlined}
\end{equation}

The models are compared using \textbf{model selection}\cite{wit2012all,zucchini2000introduction} \inlineequation[model-selection]{L(R(D(A)))} (e.g. adjusted $r^2$, AIC, ANOVA, cross-validation, etc.).

The \textbf{best fitting model} $M(R(D(A)))$ is selected for each action.

Next, the model $M(R(D(A)))$ is used to predict each $B(A(t_F))$.

Finally, \textbf{modeled value-based prioritization} \inlineequation[modeled-vbp]{Z(t_F)} is simply \eqref{vbp} with $t_F$.

\section{Choosing Meaningful Work}

The following example applies modeled value based prioritization \eqref{modeled-vbp} to the goal of choosing meaningful work. Every aspect is an example and should be reconsidered.

First, outline the parameters:

\begin{itemize}
\item \eqref{value-system} $V = $ a value system which answers ``Why work?" with ``To reduce suffering" which is defined as maximal human suffering: death\footnote{The term suffering is ambiguous because some consider that there is no suffering after death; however, the intent is to encapsulate something like the potential of life.}. Alternatives include disease burden (e.g. Quality-Adjusted Life Years [QALYs/DALYs]\cite{weinstein2009qalys}), non-human suffering, pre-birth suffering, etc.
\item \eqref{goal} $G(t) = $ eliminate human death.
\item \eqref{potential-actions} $A(t) = $ the set of actions which would eliminate human death.
\item \eqref{how-many-actions} $k = $ 2 for a single person, weighted heavily on the first item with the second item being a hedge or volunteer activity.
\item \eqref{modeled-future-time} $t_F = $ 5 years; an average amount of time under normal conditions to integrate into a new career to work on some subset of $A(t)$ (including learning, certification, building experience, networking, etc.).
\item \eqref{action-predictors} $D(A) = $ time-series data on human death by underlying cause.
\item \eqref{prediction-models} $R(D(A)) = $ linear regression with one, two, and three degrees.
\item \eqref{model-selection} $L(R(D(A))) = $ adjusted $r^2$.
\end{itemize}

$A(t)$ is the set of 176 actions which would eliminate the 176 major groups (ICD-10 sub-chapters\cite{icd10}) of underlying causes of death in the United States\cite{centers2017underlying}\textsuperscript{,}\footnote{Group Results By ``Year" And By ``ICD Sub-Chapter"; Check ``Export Results"; Uncheck ``Show Totals"}:

\begin{equation*}
\begin{gathered}
A(t) = \{\\
A_1(t) = \textrm{Eliminate: Malignant neoplasms},\\
A_2(t) = \textrm{Eliminate: Ischaemic heart diseases},\\
\textrm{\ldots}\\
A_{176}(t) = \textrm{Eliminate: Other disorders of ear}\\
\}
\end{gathered}
\end{equation*}

Review the list of actions and hypothesize scale functions. Examples:

\begin{itemize}
\item $S_1(A(t)) = 1$: Required scale function.
\item $S_2(A(t))$: Likelihood of Success. If primarily political then 0.1, else 1.
\item $S_3(A(t)) = ScaleAge(A(t))$: Scale towards younger people as they have more to lose.\newline $ScaleAge(A(t)) = \left(1 - \frac{AvgAge(A(t))}{MaxAge}\right)$
\end{itemize}

Create a table listing all actions as rows and all scale functions as columns, and fill any non-$1$ values. For example:

\begin{table}[htb]
\centering
\begin{tabular}{cccc}
\toprule
Action & $S_1$  & \ldots & $S_N$  \\
\midrule
$A_1$  & 0.1    &        &        \\
$A_2$  &        &        & 0.25   \\
\ldots &        &        &        \\
$A_N$  & 0.99   &        & 0.9    \\
\bottomrule
\end{tabular}
\caption{Example scale function table}
\label{table:scaletable}
\end{table}

$D(A)$ for each action is the time-series data of number of deaths per year. For example, for $A_1(t)$:

\begin{table}[htb]
\centering
\begin{tabular}{ccc}
\toprule
Year    & Deaths \\
\midrule
1999    & 549838 \\
\ldots  & \ldots \\
2017    & 599108 \\
\bottomrule
\end{tabular}
\caption{Deaths per year for $A_1(t)$: Malignant neoplasms}
\label{table:daa1}
\end{table}

%%%%%%%%%%%%%%
% References %
%%%%%%%%%%%%%%

\nocite{*}
\bibliography{value_based_prioritization}

\end{document}

% Create PDF on Linux:
% FILE=value_based_prioritization; pkill -9 -f ${FILE} &>/dev/null; rm -f ${FILE}*aux ${FILE}*bbl ${FILE}*bib ${FILE}*blg ${FILE}*log ${FILE}*out ${FILE}*pdf &>/dev/null; pdflatex -halt-on-error ${FILE}; bibtex ${FILE} && pdflatex ${FILE} && pdflatex ${FILE} && (xdg-open ${FILE}.pdf &)
